%
% 6.006 problem set 0 solutions template
%
\documentclass[12pt,twoside]{article}

\input{macros-sp20}
\newcommand{\theproblemsetnum}{0}

\title{6.006 Problem Set 0}

\begin{document}

\handout{Problem Set \theproblemsetnum}

\setlength{\parindent}{0pt}
\medskip\hrulefill\medskip

{\bf Name:} Kim Heesuk

\medskip\hrulefill

%%%%%%%%%%%%%%%%%%%%%%%%%%%%%%%%%%%%%%%%%%%%%%%%%%%%%
% See below for common and useful latex constructs. %
%%%%%%%%%%%%%%%%%%%%%%%%%%%%%%%%%%%%%%%%%%%%%%%%%%%%%

% Some useful commands:
% $f(x) = \Theta(x)$
% $T(x, y) \leq \log(x) + 2^y + \binom{2n}{n}$
% \ttt{code\_function}


% You can create unnumbered lists as follows:
% \begin{itemize}
%     \item First item in a list
%         \begin{itemize}
%             \item First item in a list
%                 \begin{itemize}
%                     \item First item in a list
%                     \item Second item in a list
%                 \end{itemize}
%             \item Second item in a list
%         \end{itemize}
%     \item Second item in a list
% \end{itemize}

% You can create numbered lists as follows:
% \begin{enumerate}
%     \item First item in a list
%     \item Second item in a list
%     \item Third item in a list
% \end{enumerate}

% You can write aligned equations as follows:
% \begin{align}
%     \begin{split}
%         (x+y)^3 &= (x+y)^2(x+y) \\
%                 &= (x^2+2xy+y^2)(x+y) \\
%                 &= (x^3+2x^2y+xy^2) + (x^2y+2xy^2+y^3) \\
%                 &= x^3+3x^2y+3xy^2+y^3
%     \end{split}
% \end{align}

% You can create grids/matrices as follows:
% \begin{align}
%     A =
%     \begin{bmatrix}
%         A_{11} & A_{21} \\
%         A_{21} & A_{22}
%     \end{bmatrix}
% \end{align}

\begin{problems}

\problem  % Problem 1

\begin{problemparts}
  \problempart % Problem 1a
  $A = \{1, 6, 12, 13, 9\}$, $B = \{3, 6, 12, 15\}$, $A \cap B = \{6, 12\}$.
  \problempart % Problem 1b
  $|A \cup B| = |\{1, 3, 6, 12, 13, 15, 9\}| = 7$
  \problempart % Problem 1c
  $|A - B| = |\{1, 13, 9\}| = 3$
\end{problemparts}

\problem  % Problem 2

\begin{problemparts}
  \problempart % Problem 2a
  $E[X] = 3/2$
  \problempart % Problem 2b
  $E[Y] = (7/2)^2 = 49/4$
  \problempart % Problem 2c
  $E[X + Y] = E[X] + E[Y] = 55/4$
\end{problemparts}

\problem  % Problem 3

\begin{problemparts}
  \problempart % Problem 3a
  true
  \problempart % Problem 3b
  false
  \problempart % Problem 3c
  false
\end{problemparts}

\problem  % Problem 4

For $n = 1$, $\sum_{i=1}^1 i^3 = 1 = \left[\frac{1\cdot2}{2}\right]^2$.

Assume $\sum_{i=1}^n i^3 = \left[\frac{n(n+1)}{2}\right]^2$, then
\begin{align}
    \begin{split}
      \sum_{i=1}^{n+1} i^3 &= \left[\frac{n(n+1)}{2}\right]^2 + (n+1)^3 \\
                           &= (n+1)^2\left[\left(\frac{n}{2}\right)^2 + (n+1)\right] \\
                           &= \left[\frac{(n+1)(n+2)}{2}\right]^2.
    \end{split}
\end{align}

\newpage
\problem  % Problem 5

For $|V| = 1$, given graph $G$ is acyclic since there is no edge. Let
$G_n$ be set of all connected undirected graph $G = (V, E)$ for which
$|V| = |E| + 1 = n$. Assume $(\forall G\in G_n)$ G is acyclic. Then
$(\forall G\in G_{n+1}, \exists G'\subset G)\ G'\in G_n$ since $G$ is connected.
Moreover, $G - G'$ has only one edge and one vertex that is connected to some
vertex of $G'$ since $G'$ consume $n-1$ edges. Thus $G$ is acyclic.

\problem  % Problem 6

\begin{lstlisting}
def count_long_subarray(A):
    '''
    Input:  A     | Python Tuple of positive integers
    Output: count | number of longest increasing subarrays of A
    '''
    count = 0
    length = 0
    lengths = []
    prev = 0
    for a in A:
        if a < prev:
            lengths.append(length)
            length = 1
        else:
            length += 1
        prev = a
    lengths.append(length)
    
    return lengths.count(max(lengths))

\end{lstlisting}

\end{problems}

\end{document}
