%
% 6.006 problem set 1 solutions template
%
\documentclass[12pt,twoside]{article}

\input{macros-sp20}
\newcommand{\theproblemsetnum}{1}

\title{6.006 Problem Set 1}

\begin{document}

\handout{Problem Set \theproblemsetnum}

\setlength{\parindent}{0pt}
\medskip\hrulefill\medskip

{\bf Name:} Kim Heesuk

\medskip\hrulefill

%%%%%%%%%%%%%%%%%%%%%%%%%%%%%%%%%%%%%%%%%%%%%%%%%%%%%
% See below for common and useful latex constructs. %
%%%%%%%%%%%%%%%%%%%%%%%%%%%%%%%%%%%%%%%%%%%%%%%%%%%%%

% Some useful commands:
%$f(x) = \Theta(x)$
%$T(x, y) \leq \log(x) + 2^y + \binom{2n}{n}$
% {\tt code\_function}


% You can create unnumbered lists as follows:
%\begin{itemize}
%    \item First item in a list
%        \begin{itemize}
%            \item First item in a list
%                \begin{itemize}
%                    \item First item in a list
%                    \item Second item in a list
%                \end{itemize}
%            \item Second item in a list
%        \end{itemize}
%    \item Second item in a list
%\end{itemize}

% You can create numbered lists as follows:
%\begin{enumerate}
%    \item First item in a list
%    \item Second item in a list
%    \item Third item in a list
%\end{enumerate}

% You can write aligned equations as follows:
%\begin{align}
%    \begin{split}
%        (x+y)^3 &= (x+y)^2(x+y) \\
%                &= (x^2+2xy+y^2)(x+y) \\
%                &= (x^3+2x^2y+xy^2) + (x^2y+2xy^2+y^3) \\
%                &= x^3+3x^2y+3xy^2+y^3
%    \end{split}
%\end{align}

% You can create grids/matrices as follows:
%\begin{align}
%    A =
%    \begin{bmatrix}
%        A_{11} & A_{21} \\
%        A_{21} & A_{22}
%    \end{bmatrix}
%\end{align}

% You can include images and PDFs as follows:
% \includegraphics[width=0.5\textwidth]{img.jpg}

\begin{problems}

\problem  % Problem 1

\begin{problemparts}
  \problempart % Problem 1a
  $f_1 = \log{n^n} = n\log{n} \in O(f_2)$,
  $f_3 = \log{n^{6006}} = 6006\log{n} \in \Theta(\log{n})$. So that
  $f_3 \in O(f_1)$. Meanwhile $f_4 \in O(f_2)$, and $f_5 \in O(f_3)$, the
  answer is $(f_5, f_3, f_1, f_4, f_2)$.

  \problempart % Problem 1b
  Since $n \in O(2^n)$, $f_1 \in O(f_3)$. Also
  $n \in O(n^2), n^2 \in O(2^n)$,
  $f_1 = 2^n \in \Theta(6006^n) = \Theta(f_2)$ and
  $f_3 = 2^{6006^n} \in \Theta(6006^{2^n}) = \Theta(f_4)$, the answer is
  $(\{f_1, f_2\}, f_5, \{f_3, f_4\})$.
  
  \problempart % Problem 1c
  Firstly, it is known that $\log n! \in \Theta(n\log n)$. By this,
  \begin{align}
    \begin{split}
      \log f_3 = \log((6n)!) \in \Theta(6n\log{6n}) &= \Theta(n(\log 6 + \log n)) \\
                                                    &= \Theta(n\log n).
    \end{split}
  \end{align}
  For $f_2$,
  \begin{align}
    \begin{split}
      \log f_2 = \log\choose{n}{n-6} &= \log\frac{n!}{6!(n-6)!} \\
                                     &\in \Theta\left(\log\frac{n!}{(n-6)!}\right) \\
                                     &= \Theta(\log n + \log(n-1) + \dots + \log(n-5)) \\
                                     &= \Theta(\log n).
    \end{split}
  \end{align}
  And for $f_4$,
  \begin{align}
    \begin{split}
      \log f_4 = \log\choose{n}{n/6} &= \log\frac{n!}{(n/6)!(5n/6)!} \\
                                     &\sim\log\frac{\sqrt{2\pi n}(n/e)^n}{\sqrt{2\pi n}(n/6e)^{n/6}\sqrt{2\pi n}(5n/6e)^{5n/6}} \\
                                     &= \log\frac{(6/5^{5/6})^n}{\sqrt{2\pi n}} \\
                                     &\in \Theta\left(n\log(6/5^{5/6}) - \frac{1}{2}(\log n + \log 2\pi)\right) = \Theta(n)
    \end{split}
  \end{align}
  since $6/5^{5/6} > 1$. Lastly,
  $\log f_1 = \log{n^n} \in\Theta(n\log n)$, and
  $\log f_5 = \log n^6 \in\Theta(\log n)$.
  Now we can conclude that the answer is $(\{f_2, f_5\}, f_4, \{f_1, f_3\})$.

  \problempart % Problem 1d
  
\end{problemparts}

\newpage
\problem  % Problem 2

\begin{problemparts}
\problempart % Problem 2a
\problempart % Problem 2b
\end{problemparts}

\newpage
\problem  % Problem 3

\newpage
\problem  % Problem 4

\begin{problemparts}
\problempart % Problem 4a
\problempart % Problem 4b
\problempart % Problem 4c
\problempart Submit your implementation to {\small\url{alg.mit.edu}}.
\end{problemparts}

\end{problems}

\end{document}
