%
% 6.006 problem set 1 solutions template
%
\documentclass[12pt,twoside]{article}

\input{macros-sp20}
\newcommand{\theproblemsetnum}{1}

\title{6.006 Problem Set 1}

\begin{document}

\handout{Problem Set \theproblemsetnum}

\setlength{\parindent}{0pt}
\medskip\hrulefill\medskip

{\bf Name:} Kim Heesuk

\medskip\hrulefill

%%%%%%%%%%%%%%%%%%%%%%%%%%%%%%%%%%%%%%%%%%%%%%%%%%%%%
% See below for common and useful latex constructs. %
%%%%%%%%%%%%%%%%%%%%%%%%%%%%%%%%%%%%%%%%%%%%%%%%%%%%%

% Some useful commands:
%$f(x) = \Theta(x)$
%$T(x, y) \leq \log(x) + 2^y + \binom{2n}{n}$
% {\tt code\_function}


% You can create unnumbered lists as follows:
%\begin{itemize}
%    \item First item in a list
%        \begin{itemize}
%            \item First item in a list
%                \begin{itemize}
%                    \item First item in a list
%                    \item Second item in a list
%                \end{itemize}
%            \item Second item in a list
%        \end{itemize}
%    \item Second item in a list
%\end{itemize}

% You can create numbered lists as follows:
%\begin{enumerate}
%    \item First item in a list
%    \item Second item in a list
%    \item Third item in a list
%\end{enumerate}

% You can write aligned equations as follows:
%\begin{align}
%    \begin{split}
%        (x+y)^3 &= (x+y)^2(x+y) \\
%                &= (x^2+2xy+y^2)(x+y) \\
%                &= (x^3+2x^2y+xy^2) + (x^2y+2xy^2+y^3) \\
%                &= x^3+3x^2y+3xy^2+y^3
%    \end{split}
%\end{align}

% You can create grids/matrices as follows:
%\begin{align}
%    A =
%    \begin{bmatrix}
%        A_{11} & A_{21} \\
%        A_{21} & A_{22}
%    \end{bmatrix}
%\end{align}

% You can include images and PDFs as follows:
% \includegraphics[width=0.5\textwidth]{img.jpg}

\begin{problems}

\problem  % Problem 1

\begin{problemparts}
  \problempart % Problem 1a
  $f_1 = \log{n^n} = n\log{n} \in O(f_2)$,
  $f_3 = \log{n^{6006}} = 6006\log{n} \in \Theta(\log{n})$. So that
  $f_3 \in O(f_1)$. Meanwhile $f_4 \in O(f_2)$, and $f_5 \in O(f_3)$, the
  answer is $(f_5, f_3, f_1, f_4, f_2)$.

  \problempart % Problem 1b
  $f_2 \in O(f_5)$ and $f_5 \in O(f_4)$. Also $f_1 \in O(f_2)$,
  $f_5 \in O(f_3)$.
  For $f_3$ and $f_4$,
  \begin{align}
    \begin{split}
      \log\frac{f_3}{f_4} &= \log\frac{2^{6006^n}}{6006^{2^n}} \\
                          &= 6006^n\log 2 - 2^n\log 6006 \\
                          &= 2^n(3003^n\log 2 - \log 6006) \rightarrow\infty
    \end{split}
  \end{align}
  $f_4\in O(f_3)$ and the answer is $(f_1, f_2, f_5, f_4, f_3)$.
  
  \problempart % Problem 1c
  Firstly,
  \begin{align}
    \begin{split}
      f_2 = \choose{n}{n-6} &= \frac{n!}{6!(n-6)!} \\
                            &\in \Theta\left(n(n-1)\dots (n-\5)\right) \\
                            &= \Theta(n^6),
    \end{split}
  \end{align}
  so $f_5 = n^6 \in\Theta(f_2)$. For $f_4$, by the Stirling's approximation,
  \begin{align}
    \begin{split}
      \log f_4 = \log\choose{n}{n/6} &= \log\frac{n!}{(n/6)!(5n/6)!} \\
                                     &\sim\log\frac{\sqrt{2\pi n}(n/e)^n}{\sqrt{2\pi n}(n/6e)^{n/6}\sqrt{2\pi n}(5n/6e)^{5n/6}} \\
                                     &= \log\frac{(6/5^{5/6})^n}{\sqrt{2\pi n}} \\
                                     &\in \Theta\left(n\log(6/5^{5/6}) - \frac{1}{2}(\log n + \log 2\pi)\right) = \Theta(n).
    \end{split}
  \end{align}
  since $6/5^{5/6} > 1$. Hence $f_2 (\in \Theta(f_5)) \in O(f_4)$,
  $f_4 \in O(f_1)$ since
  $f_4 \in\left\{2^p|p\in\Theta(n)\right\}\subset\Omega(n^6)$ and
  $\log f_1 \in \Theta(n\log n)$.  Also by the Stirling's
  approximation,
  \begin{align}
    \begin{split}
    f_3 = (6n)! &\sim \sqrt{12\pi n}(6n/e)^{6n} \\
                &\in \Theta((6n)^{6n}) \subset\Omega(n^n) = \Omega(f_1).
    \end{split}
  \end{align}
  Thus the answer is $(\{f_2, f_5\}, f_4, f_1, f_3)$.

  \problempart % Problem 1d
  Using the Stirling's approximation,
  $f_1 \sim n^{n+4} + \sqrt{2\pi n}(n/e)^n \in\Theta(n^{n+4})$.
  Also considering $f_5/n^{12} = n^{1/n}\rightarrow 1$,
  $f_5 = n^{12+1/n}\sim n^{12}\in\Theta(n^{12})$.
  It is obvious that $f_2=n^{7\sqrt{n}}\in O(n^{n+4}) = O(f_1)$ and $f_3 = 4^{3n\log n} \in O(7^{n^2}) = O(f_4)$.
  Finally for $f_3$ and $n^{n+4} \in\Omega(f_1)$,
  \begin{align}
    \begin{split}
      \log\frac{f_3}{n^{n+4}} &= \log\frac{4^{3n\log n}}{n^{n+4}} \\
                              &= 3n\log n \cdot\log 4 - (n+4)\log n \\
                              &= ((3\log 4 - 1)n - 4)\log n \rightarrow\infty,
    \end{split}
  \end{align}
  $f_3 \in O(n^{n+4}) = O(f_1)$. The order should be $(f_5, f_2, f_1, f_3, f_4)$.
  
\end{problemparts}

\newpage
\problem  % Problem 2

\begin{problemparts}
  \problempart % Problem 2a
  $T(k) = O(\log n) + T(k - 1)$, $T(k) = O(k\log n)$
  \begin{lstlisting}
    def reverse(D, i, k): # T(k)
      if k < 2:
        return
      D.insert_at(i + k - 1, D.delete_at(i))     # O(log(n))
      reverse(D, i, k - 1)                       # T(k - 1)
  \end{lstlisting}
  If we call \texttt{reverse(D, i, 1)}, it returns and it is correct.
  Assume \texttt{reverse(D, i, k)} works right, \texttt{reverse(D, i,
    k + 1)} should work right because it deletes element of $i$'th
  index, and insert it to $i + k$ (next to the pre-deletion index
  $i + k$) and reverses $k$ items starting at index $i$ of \texttt{D},
  which resulting reverse $k+1$ items starting at index $i$.
  
  \problempart % Problem 2b
  $T(k) = O(\log n) + O(\log n) + T(k-1)$, $T(k) = O(k\log n)$
  
  \begin{lstlisting}
    def move(D, i, k, j):          #T(k)
      if k < 1:
        return
      temp = D.delete_at(i)        # O(log(n))
      if j < i:
        D.insert_at(j + 1, temp)   # O(log(n))
      else:
        D.insert_at(j, temp)       # O(log(n))
      move(D, i, k - 1, j)         # T(k-1)
  \end{lstlisting}
    
  Note that when $j\geq i + k$ (which can be just $\neg(j < i)$ since
  $i\leq j < i+k$ is false), \texttt{D.delete\_at(i)} makes
  the index of element with pre-deletion index $j$ be $j-1$.
  
\end{problemparts}

\newpage
\problem  % Problem 3


\newpage
\problem  % Problem 4

\begin{problemparts}
\problempart % Problem 4a
\problempart % Problem 4b
\problempart % Problem 4c
\problempart Submit your implementation to {\small\url{alg.mit.edu}}.
\end{problemparts}

\end{problems}

\end{document}
